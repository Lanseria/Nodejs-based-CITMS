%%================================================
%% Filename: app01.tex
%% Encoding: UTF-8
%% Author: Yuan Xiaoshuai - yxshuai@gmail.com
%% Created: 2012-04-25 15:16
%% Last modified: 2016-08-28 21:06
%%================================================
\newgeometry{left=2cm,top=3cm,bottom=2cm,right=2cm,includefoot}
\chapter{表格附件}
\section*{\centering\sanhao\hei 毕业设计(论文)任务书}
\addcontentsline{toc}{section}{附表~1:毕业设计(论文)任务书}
\vspace*{.5ex}
\begin{center}
\renewcommand\arraystretch{1.25}
\newcommand{\minitab}[2][>{\rm}m{2em}]{\begin{tabular}{#1}#2\end{tabular}}
\begin{tabularx}{\textwidth}{|>{\centering\rm}m{2em}|X|}
\hline
\multirow{6}{2em}{\minitab{课\\题\\情\\况}} &
\minitab[>{\centering\rm}m{7em}|c]{课题名称 & } \\ 
\cline{2-2}
&\minitab[>{\centering\rm}m{7em}|>{\centering}m{3em}|>{\centering\rm}m{2em}|>{\centering}m{2em}|>{\centering\rm}m{2em}|>{\centering}m{2em}|>{\centering\rm}m{3em}|X]{
教师姓名 & & 职称 & & 学位 & & 教研室 & }\\
\cline{2-2}
 & \minitab[>{\centering\rm}m{7em}|>{\rm}c]{课题来源 & A. 科研 \checkmark\quad B. 生产\quad C. 教学\quad D. 其它}\\
\cline{2-2}
 &\minitab[>{\centering\rm}m{7em}|>{\centering\rm}m{12em}|>{\centering\rm}m{4em}|X]{课题类别 & A. 设计\quad B. 论文 \checkmark & 实施地点 & }\\
\cline{2-2}
 &
\minitab[>{\centering\rm}m{7em}|>{\centering}m{12em}|>{\centering\rm}m{4em}|>{\rm}X]{起止时间 & 2011 年 11 月 $\sim$ 2012 年 6 月 & 上机时数 & }\\
\cline{2-2}
 & \minitab[>{\centering\rm}m{7em}|>{\rm}X]{拟指导的学生数 & \minitab[c|>{\rm}c|c]{1 人 & 对学生的特殊要求 & 基础知识扎实,动手能力较强}}\\
\hline
{\renewcommand\arraystretch{1}\minitab{主\\要\\研\\究\\内\\容}}&{
\vspace*{-7ex}\begin{minipage}[t][10ex][s]{.9\textwidth}\hspace{2em}
主要研究内容
\end{minipage}
}\\
\hline
{\renewcommand\arraystretch{1}\minitab{目\\标\\和\\要\\求}} &{
\vspace*{-6.4ex}\begin{minipage}[t][10ex][s]{.9\textwidth}\hspace{2em}
目标和要求
\end{minipage}
}\\
\hline
{\renewcommand\arraystretch{1}\minitab{特\\色}} & 特色\\
\hline
{成果形式} & 论文\\
\hline
{成果价值} & 有一定的学术价值\\
\hline
{\renewcommand\arraystretch{1}\minitab{科\\室\\审\\题\\意\\见}} &
{\rm\vspace*{2.4ex}\hbox{\hspace{.2\textwidth}负责人签字:\hspace{8em}年\hspace{2em}月\hspace{2em}日}}
\\
\hline
{\renewcommand\arraystretch{1}\minitab{学\\院\\审\\批\\意\\见}} &
{\rm\vspace*{2.4ex}\hbox{\hspace{.2\textwidth}主管院长签字:\hspace{7em}年\hspace{2em}月\hspace{2em}日}}
\\
\hline
\end{tabularx}
\end{center}

\newpage
\renewcommand\arraystretch{1.25}
\section*{\centering\song\sanhao 郑州大学毕业论文开题报告表}
\addcontentsline{toc}{section}{附表~2:郑州大学毕业论文开题报告表}
\begin{center}
\vspace*{0.5ex}
\hfil 院(系):\texttt{材料科学与工程学院}\hspace{15em} 类别:论文\hfil\hfil
\vspace*{0.5ex}

\begin{tabularx}{\textwidth}{|>{\centering\rm}p{.12\textwidth}|>{\centering}p{.12\textwidth}|>{\centering\rm}p{.08\textwidth}|>{\centering}p{.15\textwidth}|>{\centering\rm}p{.05\textwidth}|Z|}
\hline
课题名称 & \multicolumn{5}{c|}{} \\
\hline
导师姓名 & & 职\quad{}称 & \multicolumn{3}{c|}{} \\
\hline
学生姓名 & & 学\quad{}号 & & 专业 & \\
\hline
\multicolumn{6}{|c|}{
\begin{minipage}[c]{.96\textwidth}
\vskip 1ex
\textrm{开题报告内容:}(调研资料的准备,选题依据、目的、要求;毕业论文进度安
排;完成毕业论文所需要实验条件等、主要参考文献与资料情况等)\CJKindent

\vspace*{.5ex}
一、选题依据、目的和要求
\vspace*{.5ex}

\vspace*{15ex}

\vspace*{.5ex}
二、实验和检测方法
\vspace*{.5ex}

\vspace*{15ex}

\vspace*{.5ex}
三、实验进度安排
\vspace*{.5ex}

\vspace*{15ex}

\vspace*{.5ex}
四、主要参考文献
\vspace*{.5ex}

\vspace*{15ex}

% \begin{enumerate}[{$[$}1{$]$}]
% 
% \item 
% 
% \end{enumerate}

\vskip 5ex
\hfill \textrm{指导教师:}\hspace{.4\textwidth}
\vskip 1.5ex
\hfill \textrm{学\hspace{2em}生:}\hspace{.4\textwidth}
\vskip 1.5ex
\hfill \textrm{年\hspace{2em}月\hspace{2em}日}\hspace{.2\textwidth}
\vspace{1ex}
\end{minipage}}\\
\hline
\end{tabularx}
\end{center}

\newpage

\section*{\centering\sanhao 毕业设计(论文)计划进程表}
\addcontentsline{toc}{section}{附表~3:毕业设计(论文)计划进程表}

\begin{center}
\renewcommand\arraystretch{1.5}
\newcommand{\minitab}[2][>{\rm}c]{\begin{tabular}{#1}#2\end{tabular}}
\begin{tabularx}{\textwidth}{|>{\centering\rm}p{.12\textwidth}|X|}
\hline
学生姓名 & \minitab[c|>{\rm}c|c|>{\rm}c|c|>{\rm}c|c]{\hspace{4em} & 学号 & \hspace{4em} & 教师姓名 & \hspace{4em} & 职称 & \hspace{3em}}\\\hline
题\hspace{2em}目 & {}\\\hline
周\hspace{2em}次 & {\hfil\textrm{工作内容}\hfil}\\\hline
 & \\
1 $\sim$ 8 周 & \\
9 $\sim$ 10 周 & \\
11 $\sim$ 16 周 & \\
17 $\sim$ 18 周 & \\
19 $\sim$ 21 周 & \\
22 $\sim$ 23 周 & \\
24 周 & \\
 & \begin{minipage}[c]{\textwidth}
\vspace{64ex}
\end{minipage}\\\hline
\end{tabularx}
\end{center}

\newpage
\section*{\centering\sanhao 郑州大学毕业设计中期检查表}
\addcontentsline{toc}{section}{附表~4:郑州大学毕业设计中期检查表}
\begin{center}
\vspace*{0.5ex}
院系:\texttt{材料科学与工程学院}\hfill
\vspace*{0.5ex}

\newcommand{\minitab}[2][>{\rm}c]{\begin{tabular}{#1}#2\end{tabular}}
\begin{tabularx}{\textwidth}{|l|}
\hline
\minitab[>{\rm}c|X]{题\hspace{2em}目 & }\\\hline
\minitab[>{\rm}c|c|>{\rm}c|c]{导师姓名 & \hspace{3em} & 职称 & \hspace{3em}}\\\hline
\minitab[>{\rm}c|c|>{\rm}c|c|>{\rm}c|c]{学生姓名 & \hspace{3em} & 学号 & \hspace{5em} 
& 专业 & \hspace{6em}}\\\hline
\begin{minipage}[c]{.96\textwidth}
\vskip 2ex 
\textrm{一、阶段性成果}
\vskip 1ex\CJKindent

\vspace*{15ex}

\vspace{20ex}
\end{minipage}\\\hline
\begin{minipage}[c]{.96\textwidth}
\vskip 2ex
\textrm{二、存在问题及解决方法}
\vskip 1ex\CJKindent

\vspace*{15ex}

\vskip 1ex
解决方法:
\vskip 1ex

\vskip 20ex
\hfill \textrm{指导教师:}\hspace{.4\textwidth}
\vskip 1.5ex
\hfill \textrm{学\hspace{2em}生:}\hspace{.4\textwidth}
\vskip 1.5ex
\hfill \textrm{年\hspace{2em}月\hspace{2em}日}\hspace{.2\textwidth}
\vspace{2ex}
\end{minipage}\\\hline
\end{tabularx}
\end{center}

\newpage
\section*{\centering\sanhao 毕业设计(论文)成绩评定表}
\addcontentsline{toc}{section}{附表~5:毕业设计(论文)成绩评定表}
\begin{center}
\vspace*{.5ex}
\hspace{4em}学院:\texttt{材料科学与工程学院}\hfill 班级:\hspace{6em}
\vspace*{1ex}

\renewcommand\arraystretch{1.5}
\newcommand{\minitab}[2][>{\rm}c]{\begin{tabular}{#1}#2\end{tabular}}
\begin{tabularx}{\textwidth}{!{\vrule width1.5bp}>{\centering\rm}Z|>{\centering}p{6em}|>{\rm}Z|>{\centering}p{8em}|>{\rm}Z|>{\centering}p{5em}!{\vrule width1.5bp}}
\Xhline{1.5bp}
姓名 & & 学号 & & 总成绩 & \\\hline
题目 & \multicolumn{5}{c!{\vrule width1.5bp}}{}\\\hline
\multirow{8}{3em}{% \renewcommand\arraystretch{1}
  \minitab{指\\导\\老\\师\\评\\语}} &
  \multicolumn{5}{>{\rm}c!{\vrule width1.5bp}}{\multirow{5}*{}}\\
  & \multicolumn{5}{>{\rm}c!{\vrule width1.5bp}}{}\\
  & \multicolumn{5}{>{\rm}c!{\vrule width1.5bp}}{}\\
  & \multicolumn{5}{>{\rm}c!{\vrule width1.5bp}}{}\\
  & \multicolumn{5}{>{\rm}c!{\vrule width1.5bp}}{}\\
  & \multicolumn{5}{>{\rm}c!{\vrule width1.5bp}}{}\\
  & \multicolumn{5}{>{\rm}c!{\vrule width1.5bp}}{}\\
  \cline{2-6}
  & \multicolumn{5}{>{\rm}c!{\vrule width1.5bp}}{
\hspace{2em}评定成绩:\hfill 签名:\hfill \hspace{2em}年\hspace{2em}月
\hspace{2em}日\hspace{2em}
}\\
\hline
\multirow{7}{3em}{% \renewcommand\arraystretch{1}
  \minitab{评\\阅\\人\\评\\语}} &
  \multicolumn{5}{>{\rm}c!{\vrule width1.5bp}}{\multirow{5}*{}}\\
  & \multicolumn{5}{>{\rm}c!{\vrule width1.5bp}}{}\\
  & \multicolumn{5}{>{\rm}c!{\vrule width1.5bp}}{}\\
  & \multicolumn{5}{>{\rm}c!{\vrule width1.5bp}}{}\\
  & \multicolumn{5}{>{\rm}c!{\vrule width1.5bp}}{}\\
  & \multicolumn{5}{>{\rm}c!{\vrule width1.5bp}}{}\\
  \cline{2-6}
  & \multicolumn{5}{>{\rm}c!{\vrule width1.5bp}}{
\hspace{2em}评定成绩:\hfill 签名:\hfill \hspace{2em}年\hspace{2em}月
\hspace{2em}日\hspace{2em}
}\\
\hline
\multirow{8}{3em}{% \renewcommand\arraystretch{1}
  \minitab{答\\辩\\小\\组\\评\\语}} &
  \multicolumn{5}{>{\rm}c!{\vrule width1.5bp}}{\multirow{5}*{}}\\
  & \multicolumn{5}{>{\rm}c!{\vrule width1.5bp}}{}\\
  & \multicolumn{5}{>{\rm}c!{\vrule width1.5bp}}{}\\
  & \multicolumn{5}{>{\rm}c!{\vrule width1.5bp}}{}\\
  & \multicolumn{5}{>{\rm}c!{\vrule width1.5bp}}{}\\
  & \multicolumn{5}{>{\rm}c!{\vrule width1.5bp}}{答辩组成员签名\hspace{12em}}\\
  & \multicolumn{5}{>{\rm}c!{\vrule width1.5bp}}{}\\
  \cline{2-6}
  & \multicolumn{5}{>{\rm}c!{\vrule width1.5bp}}{
\hspace{2em}评定成绩:\hfill 签名:\hfill \hspace{2em}年\hspace{2em}月
\hspace{2em}日\hspace{2em}
}\\
\Xhline{1.5bp}
\end{tabularx}
\end{center}
{\wuhao 注:设计(论文)总成绩 = 指导教师评定成绩(30\%) + 评阅人评定成绩(
30\%) +答辩成绩(40\%)}
\restoregeometry
